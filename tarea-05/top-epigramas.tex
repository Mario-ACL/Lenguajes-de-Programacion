\documentclass{article}
\usepackage{graphicx} % Required for inserting images

\title{Tarea-05 top-epigramas}
\author{Mario Alejandro Castro Lerma}
\date{Octubre 2023}

\begin{document}

\maketitle

\section{Epigramas que mas me gustaron:}
\begin{enumerate}
    \item Programming is an unnatural act.
    
    Para mi significa que la accion de programar es algo que no es natural del ser humano, es una forma de pensar diferente a la naturaleza del ser humano.
    
    Lo elegí por que me parece gracioso verlo de una forma mas literal ya que podríamos entenderlo como que los programadores somos como magos arcanos que cada vez que programamos estamos haciendo uso de poderes sobrenaturales.

    \item Optimization hinders evolution.

    Para mi significa que por el buscar optimizar todo, perdemos la búsqueda de nuevas formas de encontrar soluciones de forma que no evolucionamos.
    
    Este epigrama me gusto por que indica como el optimizar las cosas nos lleva al descarte de soluciones que pensamos que no son óptimas o que tal vez no son muy convencionales por que estamos en busca de hacer las cosas mas rápido, por lo tanto no probamos nuevas formas de solucionar las cosas y por ende no evolucionamos.

    \item In software systems, it is often the early bird that makes the worm.

    El que se levanta primero es el que tiene crea el gusano, es decir que aquellos que trabajan temprano suelen ser los que crean los programas primero.
    
    Este epigrama me gusto por que indica como en el mundo del software, los que trabajan desde temprano suelen ser los que crean los programas importantes, es decir, solamente puedes crear un programa original y popular en el mercado si eres el primero que lo crea.

    \item Like punning, programming is a play on words.
    
    Este epigrama intenta decir que como la palabra "punning", "programming" tambien es un juego de palabras, esto hace referencia a que programar esta hecho de pro y grammar.
    
    La verdad es que nunca había pensado en que la palabra programar se podría decir que  esta compuesta de 2 palabras. La verdad es que tiene mucho sentido la palabra ya que programar es el manejo de gramaticas de forma profesional y por ende la palabra "programar".

    \item Often it is the means that justify the ends: Goals advance technique and technique survives
    even when goal structures crumble.

    Muchas veces los medios justifican las metas, los objetivos nos ayudan a avanzar nuestra tecnica y nuestra tecnica sobrevive incluso cuando perdemos las metas.

    Este epigrama me gusto por que habla sobre como no importa realmente si no logramos nuestros objetivos, ya que sin importar que suceda con nuestro objetivo, las habilidades y experiencia que desarrollamos a lo largo del camino permanece con nosotros. Sin importar que hagamos o cuantas veces fallemos, mientras intentemos avanzar al menos caeremos hacia enfrente.
    
    \item A year spent in artificial intelligence is enough to make one believe in God

    Este epigrama dice que si pasas un año en inteligencia artificial esto es suficiente para hacerte creer en Dios es decir, esto puede significar 2 cosas, una es que vas a sufrir tanto que vas a buscar ayuda de un ente superior, y la otra es que es tan increíble que podrías pensar que solo un dios puede con esto.

    Este epigrama me pareció gracioso ya que Alan Perlis es una persona muy famosa en ciencias computacionales y que él mencione esto significa que la inteligencia artificial es algo a lo que tenerle miedo.
    
    \item The computer is the ultimate polluter: its feces are indistinguish- able from the food it
    produces.

    Las computadoras son las mayores contaminadoras, ya que sus desechos son indistinguibles de productos bien hechos por que no puedes diferenciar fácilmente si un programa es basura o no.

    Es muy interesante pensar en que las computadoras contaminan mucho por que no puedes saber si lo que se esta produciendo es correcto o incorrecto sin tomarte el tiempo en ello y es también muy fácil producir programas inútiles que se esconden entre programas correctos.
    
    \item When someone says ”I want a programming language in which I need only say what I wish done,” give him a lollipop.

    Si alguien menciona que quiere un lenguaje en el que solo tenga que mencionar sus deseos y que se cumplan, dale una paleta. Esto habla sobre la inocencia de alguien que no sabe lo necesario para poder crear programas.

    Este me gusto por que muestra como alguien que tal vez esta empezando menciona como quisiera que los lenguajes de programación sean mas simples para poder cumplir los programas que desea y se burla de su inocencia
    
    \item Don’t have good ideas if you aren’t willing to be responsible for them

    No tengas buenas ideas si no vas a hacerte responsable de ellas. Si tienes una idea, debes tráela a la realidad.

    Esta la elegi por que habla sobre como necesitas tener el impulso de hacer las ideas que tengas por que si no entonces no tiene sentido que las tengas, necesitas crear y llevar tus ideas a la realidad. No traer una buena idea a la realidad es como si nunca hubiera existido. 
    
    \item In man-machine symbiosis, it is man who must adjust: The machines can’t

    En una simbiosis entre humano y maquina, son los humanos los que se tienen que ajustar, no las maquinas. Una maquina no se puede ajustar pero un humano si.

    Esta me gusto por que las maquinas no se ajustan ya que estas dependen completamente de los humanos para ajustarse, una maquina no puede cambiar si no esta hecha por un humano para cambiar. Las maquinas no cambian sin interacción humana.


\end{enumerate}

\section{Epigramas que no me gustaron:}
\begin{enumerate}
    \item Think of it! With VLSI we can pack 100 ENIACS in 1 sq. cm.

    Habla sobre como en un chip podemos integrar 100 computadoras ENIACS en 1 cm cuadrado

    No me gusto por que siento que es algo vieja la comparación que realiza, ya que hoy en día los transistores para un chip de un CPU se mide en nanómetros lo cual es miles de veces menor a lo que esta comparando Alan Perlis, es decir esta desactualizada la comparación.

    \item Whenever two programmers meet to criticize their programs, both are silent

    Cuando 2 programadores se juntan a criticar sus programas, los 2 se quedan callados.

    Honestamente no entendí, hay ambigüedad, se refiere a que los 2 están avergonzados de sus programas o a que los programadores son anti-sociales?
    
    
    \item The debate rages on: is PL/I Bachtrian or Dromedary?

    El debate continua: PL/I es Bachtrian o dromedario?

    Honestamente en este estoy perdido, busque sobre que era Bachtrian y lo unico que me aparecia era sobre referencias a esta frase hacker. En dromedary busque pero solo me aparecieron camellos. Esta frase realmente no la entendi pero por falta de contexto.
    
    \item We will never run out of things to program as long as there is a single program around.

    Esta frase habla sobre como nunca nos quedaremos con cosas que programar mientras exista un solo programa.

    Esta la razón por la que no me gusto es que esto implica que si no existieran los programas entonces no programaríamos, pero entonces como empezamos a programar si antes no existían los programas?
    
    \item When someone says ”I want a programming language in which I need only say what I wish done,” give him a lollipop

    Si alguien menciona que quiere un lenguaje en el que solo tenga que mencionar sus deseos y que se cumplan, dale una paleta. Esto habla sobre la inocencia de alguien que no sabe lo necesario para poder crear programas.

    Aunque lo agregue a los que me habían gustado, también no me gusta por que implica que nunca llegaremos a una era donde programar sea posible con lenguaje natural, algo que realmente nos acercamos con herramientas como chatGPT, además de que se vale soñar. Ya que veo los 2 lados del argumento, decidí agregarla a las que no me gustaron también.
    
    \item If your computer speaks English, it was probably made in Japan.

    Si tu computadora habla ingles, probablemente esta hecha en japón. Ya que creo que todo lo excéntrico esta realizado en japón generalmente según esta frase.

    Esta no me gusto por que creo que hoy en día es mas realista decir que este tipo de cosas son realizadas en china, de forma que esta desactualizada la frase o algo no me quedo claro.
    
    \item Is it possible that software is not like anything else, that it is meant to be discarded: that the whole point is to see it as a soap bubble?

    Tal vez es posible que el software no es como las demás cosas, que tal vez esta hecho para ser descartado: tal vez el punto es verlo como una burbuja de jabón? Que el software no debería ser para siempre.

    Esta frase no me gusto ya que creo yo que el software debería ser preservado, por que es muy triste pensar que el software se pierda al pasar de los años y que tal vez programas brillantes sean perdidos al tiempo.
    
    \item Documentation is like term insurance: It satisfies because almost no one who subscribes to it depends on its benefits

    La documentación es como la aseguranza, es satisfacible por que casi nadie que se suscribe a ella depende de sus beneficios, no es útil para quien la sigue.

    Esta no me gusto por que realmente la documentación es útil para todos, incluso para quien hace el programa, al menos para mi, cuando pasa mucho tiempo y veo un programa mio viejo me doy cuenta de que no recuerdo como funcionan muchas cosas y la falta de documentación afecta.
    
    \item Think of all the psychic energy expended in seeking a fundamental distinction between ”algorithm” and ”program”.

    Esta frase quiere decir que es dificil distinguir entre algoritmo y programa.

    Aunque es graciosa la forma en que lo menciona, creo que es muy simple la frase.
    
    \item You can measure a programmer’s perspective by noting his attitude on the continuing vitality of FORTRAN.

    Puedes saber la perspectiva de un programador al observar su actitud con respecto a FORTRAN.

    Esta frase no se si esta intentado ofenderme de alguna forma ya que nunca he utilizado FORTRAN y por lo tanto tengo muy poca perspectiva o tal vez es bueno que no haya usado FORTRAN, de cualquier forma la frase no me gusto por que solo he visto una vez FORTRAN y no me agrado.


\end{enumerate}




\end{document}
